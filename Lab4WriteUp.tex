% Options for packages loaded elsewhere
% Options for packages loaded elsewhere
\PassOptionsToPackage{unicode}{hyperref}
\PassOptionsToPackage{hyphens}{url}
\PassOptionsToPackage{dvipsnames,svgnames,x11names}{xcolor}
%
\documentclass[
]{article}
\usepackage{xcolor}
\usepackage[top=0.75in,left=0.75in,bottom=0.75in,right=0.75in]{geometry}
\usepackage{amsmath,amssymb}
\setcounter{secnumdepth}{5}
\usepackage{iftex}
\ifPDFTeX
  \usepackage[T1]{fontenc}
  \usepackage[utf8]{inputenc}
  \usepackage{textcomp} % provide euro and other symbols
\else % if luatex or xetex
  \usepackage{unicode-math} % this also loads fontspec
  \defaultfontfeatures{Scale=MatchLowercase}
  \defaultfontfeatures[\rmfamily]{Ligatures=TeX,Scale=1}
\fi
\usepackage{lmodern}
\ifPDFTeX\else
  % xetex/luatex font selection
\fi
% Use upquote if available, for straight quotes in verbatim environments
\IfFileExists{upquote.sty}{\usepackage{upquote}}{}
\IfFileExists{microtype.sty}{% use microtype if available
  \usepackage[]{microtype}
  \UseMicrotypeSet[protrusion]{basicmath} % disable protrusion for tt fonts
}{}
\makeatletter
\@ifundefined{KOMAClassName}{% if non-KOMA class
  \IfFileExists{parskip.sty}{%
    \usepackage{parskip}
  }{% else
    \setlength{\parindent}{0pt}
    \setlength{\parskip}{6pt plus 2pt minus 1pt}}
}{% if KOMA class
  \KOMAoptions{parskip=half}}
\makeatother
% Make \paragraph and \subparagraph free-standing
\makeatletter
\ifx\paragraph\undefined\else
  \let\oldparagraph\paragraph
  \renewcommand{\paragraph}{
    \@ifstar
      \xxxParagraphStar
      \xxxParagraphNoStar
  }
  \newcommand{\xxxParagraphStar}[1]{\oldparagraph*{#1}\mbox{}}
  \newcommand{\xxxParagraphNoStar}[1]{\oldparagraph{#1}\mbox{}}
\fi
\ifx\subparagraph\undefined\else
  \let\oldsubparagraph\subparagraph
  \renewcommand{\subparagraph}{
    \@ifstar
      \xxxSubParagraphStar
      \xxxSubParagraphNoStar
  }
  \newcommand{\xxxSubParagraphStar}[1]{\oldsubparagraph*{#1}\mbox{}}
  \newcommand{\xxxSubParagraphNoStar}[1]{\oldsubparagraph{#1}\mbox{}}
\fi
\makeatother

\usepackage{color}
\usepackage{fancyvrb}
\newcommand{\VerbBar}{|}
\newcommand{\VERB}{\Verb[commandchars=\\\{\}]}
\DefineVerbatimEnvironment{Highlighting}{Verbatim}{commandchars=\\\{\}}
% Add ',fontsize=\small' for more characters per line
\usepackage{framed}
\definecolor{shadecolor}{RGB}{241,243,245}
\newenvironment{Shaded}{\begin{snugshade}}{\end{snugshade}}
\newcommand{\AlertTok}[1]{\textcolor[rgb]{0.68,0.00,0.00}{#1}}
\newcommand{\AnnotationTok}[1]{\textcolor[rgb]{0.37,0.37,0.37}{#1}}
\newcommand{\AttributeTok}[1]{\textcolor[rgb]{0.40,0.45,0.13}{#1}}
\newcommand{\BaseNTok}[1]{\textcolor[rgb]{0.68,0.00,0.00}{#1}}
\newcommand{\BuiltInTok}[1]{\textcolor[rgb]{0.00,0.23,0.31}{#1}}
\newcommand{\CharTok}[1]{\textcolor[rgb]{0.13,0.47,0.30}{#1}}
\newcommand{\CommentTok}[1]{\textcolor[rgb]{0.37,0.37,0.37}{#1}}
\newcommand{\CommentVarTok}[1]{\textcolor[rgb]{0.37,0.37,0.37}{\textit{#1}}}
\newcommand{\ConstantTok}[1]{\textcolor[rgb]{0.56,0.35,0.01}{#1}}
\newcommand{\ControlFlowTok}[1]{\textcolor[rgb]{0.00,0.23,0.31}{\textbf{#1}}}
\newcommand{\DataTypeTok}[1]{\textcolor[rgb]{0.68,0.00,0.00}{#1}}
\newcommand{\DecValTok}[1]{\textcolor[rgb]{0.68,0.00,0.00}{#1}}
\newcommand{\DocumentationTok}[1]{\textcolor[rgb]{0.37,0.37,0.37}{\textit{#1}}}
\newcommand{\ErrorTok}[1]{\textcolor[rgb]{0.68,0.00,0.00}{#1}}
\newcommand{\ExtensionTok}[1]{\textcolor[rgb]{0.00,0.23,0.31}{#1}}
\newcommand{\FloatTok}[1]{\textcolor[rgb]{0.68,0.00,0.00}{#1}}
\newcommand{\FunctionTok}[1]{\textcolor[rgb]{0.28,0.35,0.67}{#1}}
\newcommand{\ImportTok}[1]{\textcolor[rgb]{0.00,0.46,0.62}{#1}}
\newcommand{\InformationTok}[1]{\textcolor[rgb]{0.37,0.37,0.37}{#1}}
\newcommand{\KeywordTok}[1]{\textcolor[rgb]{0.00,0.23,0.31}{\textbf{#1}}}
\newcommand{\NormalTok}[1]{\textcolor[rgb]{0.00,0.23,0.31}{#1}}
\newcommand{\OperatorTok}[1]{\textcolor[rgb]{0.37,0.37,0.37}{#1}}
\newcommand{\OtherTok}[1]{\textcolor[rgb]{0.00,0.23,0.31}{#1}}
\newcommand{\PreprocessorTok}[1]{\textcolor[rgb]{0.68,0.00,0.00}{#1}}
\newcommand{\RegionMarkerTok}[1]{\textcolor[rgb]{0.00,0.23,0.31}{#1}}
\newcommand{\SpecialCharTok}[1]{\textcolor[rgb]{0.37,0.37,0.37}{#1}}
\newcommand{\SpecialStringTok}[1]{\textcolor[rgb]{0.13,0.47,0.30}{#1}}
\newcommand{\StringTok}[1]{\textcolor[rgb]{0.13,0.47,0.30}{#1}}
\newcommand{\VariableTok}[1]{\textcolor[rgb]{0.07,0.07,0.07}{#1}}
\newcommand{\VerbatimStringTok}[1]{\textcolor[rgb]{0.13,0.47,0.30}{#1}}
\newcommand{\WarningTok}[1]{\textcolor[rgb]{0.37,0.37,0.37}{\textit{#1}}}

\usepackage{longtable,booktabs,array}
\usepackage{calc} % for calculating minipage widths
% Correct order of tables after \paragraph or \subparagraph
\usepackage{etoolbox}
\makeatletter
\patchcmd\longtable{\par}{\if@noskipsec\mbox{}\fi\par}{}{}
\makeatother
% Allow footnotes in longtable head/foot
\IfFileExists{footnotehyper.sty}{\usepackage{footnotehyper}}{\usepackage{footnote}}
\makesavenoteenv{longtable}
\usepackage{graphicx}
\makeatletter
\newsavebox\pandoc@box
\newcommand*\pandocbounded[1]{% scales image to fit in text height/width
  \sbox\pandoc@box{#1}%
  \Gscale@div\@tempa{\textheight}{\dimexpr\ht\pandoc@box+\dp\pandoc@box\relax}%
  \Gscale@div\@tempb{\linewidth}{\wd\pandoc@box}%
  \ifdim\@tempb\p@<\@tempa\p@\let\@tempa\@tempb\fi% select the smaller of both
  \ifdim\@tempa\p@<\p@\scalebox{\@tempa}{\usebox\pandoc@box}%
  \else\usebox{\pandoc@box}%
  \fi%
}
% Set default figure placement to htbp
\def\fps@figure{htbp}
\makeatother


% definitions for citeproc citations
\NewDocumentCommand\citeproctext{}{}
\NewDocumentCommand\citeproc{mm}{%
  \begingroup\def\citeproctext{#2}\cite{#1}\endgroup}
\makeatletter
 % allow citations to break across lines
 \let\@cite@ofmt\@firstofone
 % avoid brackets around text for \cite:
 \def\@biblabel#1{}
 \def\@cite#1#2{{#1\if@tempswa , #2\fi}}
\makeatother
\newlength{\cslhangindent}
\setlength{\cslhangindent}{1.5em}
\newlength{\csllabelwidth}
\setlength{\csllabelwidth}{3em}
\newenvironment{CSLReferences}[2] % #1 hanging-indent, #2 entry-spacing
 {\begin{list}{}{%
  \setlength{\itemindent}{0pt}
  \setlength{\leftmargin}{0pt}
  \setlength{\parsep}{0pt}
  % turn on hanging indent if param 1 is 1
  \ifodd #1
   \setlength{\leftmargin}{\cslhangindent}
   \setlength{\itemindent}{-1\cslhangindent}
  \fi
  % set entry spacing
  \setlength{\itemsep}{#2\baselineskip}}}
 {\end{list}}
\usepackage{calc}
\newcommand{\CSLBlock}[1]{\hfill\break\parbox[t]{\linewidth}{\strut\ignorespaces#1\strut}}
\newcommand{\CSLLeftMargin}[1]{\parbox[t]{\csllabelwidth}{\strut#1\strut}}
\newcommand{\CSLRightInline}[1]{\parbox[t]{\linewidth - \csllabelwidth}{\strut#1\strut}}
\newcommand{\CSLIndent}[1]{\hspace{\cslhangindent}#1}



\setlength{\emergencystretch}{3em} % prevent overfull lines

\providecommand{\tightlist}{%
  \setlength{\itemsep}{0pt}\setlength{\parskip}{0pt}}



 


\makeatletter
\@ifpackageloaded{caption}{}{\usepackage{caption}}
\AtBeginDocument{%
\ifdefined\contentsname
  \renewcommand*\contentsname{Table of contents}
\else
  \newcommand\contentsname{Table of contents}
\fi
\ifdefined\listfigurename
  \renewcommand*\listfigurename{List of Figures}
\else
  \newcommand\listfigurename{List of Figures}
\fi
\ifdefined\listtablename
  \renewcommand*\listtablename{List of Tables}
\else
  \newcommand\listtablename{List of Tables}
\fi
\ifdefined\figurename
  \renewcommand*\figurename{Figure}
\else
  \newcommand\figurename{Figure}
\fi
\ifdefined\tablename
  \renewcommand*\tablename{Table}
\else
  \newcommand\tablename{Table}
\fi
}
\@ifpackageloaded{float}{}{\usepackage{float}}
\floatstyle{ruled}
\@ifundefined{c@chapter}{\newfloat{codelisting}{h}{lop}}{\newfloat{codelisting}{h}{lop}[chapter]}
\floatname{codelisting}{Listing}
\newcommand*\listoflistings{\listof{codelisting}{List of Listings}}
\makeatother
\makeatletter
\makeatother
\makeatletter
\@ifpackageloaded{caption}{}{\usepackage{caption}}
\@ifpackageloaded{subcaption}{}{\usepackage{subcaption}}
\makeatother
\usepackage{bookmark}
\IfFileExists{xurl.sty}{\usepackage{xurl}}{} % add URL line breaks if available
\urlstyle{same}
\hypersetup{
  pdftitle={Lab Four -- Programming in R},
  colorlinks=true,
  linkcolor={blue},
  filecolor={Maroon},
  citecolor={Blue},
  urlcolor={Blue},
  pdfcreator={LaTeX via pandoc}}


\title{Lab Four -- Programming in R}
\author{}
\date{}
\begin{document}
\maketitle


\normalsize

\normalsize

\begin{itemize}
\tightlist
\item
  Complete the tasks below. Make sure to start your solutions in on a
  new line that starts with ``\textbf{Solution}:''.
\item
  Make sure to use the Quarto Cheatsheet. This will make completing and
  writing up the lab \emph{much} easier.
\end{itemize}

In this lab, we will build a mealkit recipe generator. I created a small
website of about 40 recipes. We will scrape recipes from that website,
randomly select three meals, and print a grocery list with recipe cards.

\section{Preliminaries}\label{preliminaries}

\subsection{Part a}\label{part-a}

Below, I create a numeric vector filled with random observations. Ask
for the 3rd item. \textbf{Note}: The \texttt{set.seed(7272)} portion
ensures we all get the same answer.

\scriptsize

\begin{Shaded}
\begin{Highlighting}[numbers=left,,]
\FunctionTok{set.seed}\NormalTok{(}\DecValTok{7272}\NormalTok{) }\CommentTok{\# sets the randomization seed for replication}
\NormalTok{x }\OtherTok{\textless{}{-}} \FunctionTok{sample}\NormalTok{(}\AttributeTok{x=}\DecValTok{1}\SpecialCharTok{:}\DecValTok{10}\NormalTok{,          }\CommentTok{\# sample from 1, 2, 3, ..., 10 }
            \AttributeTok{size=}\DecValTok{10}\NormalTok{,         }\CommentTok{\# sample of 10}
            \AttributeTok{replace =} \ConstantTok{TRUE}\NormalTok{)  }\CommentTok{\# allowed to sample the same item multiple times}
\NormalTok{x[}\DecValTok{3}\NormalTok{] }\CommentTok{\#asking for the 3rd index}
\end{Highlighting}
\end{Shaded}

\begin{verbatim}
[1] 5
\end{verbatim}

\normalsize

\subsection{Part b}\label{part-b}

Below, I create a data frame filled with random observations.

\normalsize

\begin{Shaded}
\begin{Highlighting}[numbers=left,,]
\FunctionTok{set.seed}\NormalTok{(}\DecValTok{7272}\NormalTok{) }\CommentTok{\# sets the randomization seed for replication}
\NormalTok{df }\OtherTok{\textless{}{-}} \FunctionTok{data.frame}\NormalTok{(}\AttributeTok{x1 =} \FunctionTok{sample}\NormalTok{(}\AttributeTok{x=}\DecValTok{1}\SpecialCharTok{:}\DecValTok{10}\NormalTok{,          }\CommentTok{\# sample from 1, 2, 3, ..., 10 }
                             \AttributeTok{size=}\DecValTok{10}\NormalTok{,         }\CommentTok{\# sample of 10}
                             \AttributeTok{replace =} \ConstantTok{TRUE}\NormalTok{), }\CommentTok{\# allowed to sample the same item multiple times}
                 \AttributeTok{x2 =} \FunctionTok{sample}\NormalTok{(}\AttributeTok{x=}\DecValTok{1}\SpecialCharTok{:}\DecValTok{10}\NormalTok{,          }\CommentTok{\# sample from 1, 2, 3, ..., 10 }
                             \AttributeTok{size=}\DecValTok{10}\NormalTok{,         }\CommentTok{\# sample of 10}
                             \AttributeTok{replace =} \ConstantTok{TRUE}\NormalTok{), }\CommentTok{\# allowed to sample the same item multiple times}
                 \AttributeTok{x3 =} \FunctionTok{sample}\NormalTok{(}\AttributeTok{x=}\DecValTok{1}\SpecialCharTok{:}\DecValTok{10}\NormalTok{,          }\CommentTok{\# sample from 1, 2, 3, ..., 10 }
                             \AttributeTok{size=}\DecValTok{10}\NormalTok{,         }\CommentTok{\# sample of 10}
                             \AttributeTok{replace =} \ConstantTok{TRUE}\NormalTok{)  }\CommentTok{\# allowed to sample the same item multiple times,}
\NormalTok{)}
\end{Highlighting}
\end{Shaded}

\normalsize

\subsection{Part b}\label{part-b-1}

Use the \texttt{head(...)} function to peek at the data frame.

\normalsize

\begin{Shaded}
\begin{Highlighting}[numbers=left,,]
\FunctionTok{head}\NormalTok{(df) }
\end{Highlighting}
\end{Shaded}

\begin{verbatim}
  x1 x2 x3
1  8  1  8
2  3  7  2
3  5  3  3
4  3  6  2
5  2  2  3
6  3  8 10
\end{verbatim}

\normalsize

\subsection{Part c}\label{part-c}

Ask for the column \texttt{x1}.

\normalsize

\begin{Shaded}
\begin{Highlighting}[numbers=left,,]
\NormalTok{df[,}\DecValTok{1}\NormalTok{]}
\end{Highlighting}
\end{Shaded}

\begin{verbatim}
 [1] 8 3 5 3 2 3 1 2 6 2
\end{verbatim}

\normalsize

\subsection{Part d}\label{part-d}

Ask for the fifth row of the data frame.

\normalsize

\begin{Shaded}
\begin{Highlighting}[numbers=left,,]
\NormalTok{df[}\DecValTok{5}\NormalTok{,]}
\end{Highlighting}
\end{Shaded}

\begin{verbatim}
  x1 x2 x3
5  2  2  3
\end{verbatim}

\normalsize

\subsection{Part e}\label{part-e}

Ask for the the value of \texttt{x1} in the fifth row of the data frame.

\normalsize

\begin{Shaded}
\begin{Highlighting}[numbers=left,,]
\NormalTok{df[}\DecValTok{5}\NormalTok{,}\DecValTok{1}\NormalTok{]}
\end{Highlighting}
\end{Shaded}

\begin{verbatim}
[1] 2
\end{verbatim}

\normalsize

\subsection{Part f}\label{part-f}

Ask for the the value of in the third column and the fifth row of the
data frame.

\normalsize

\begin{Shaded}
\begin{Highlighting}[numbers=left,,]
\NormalTok{df[}\DecValTok{5}\NormalTok{,}\DecValTok{3}\NormalTok{]}
\end{Highlighting}
\end{Shaded}

\begin{verbatim}
[1] 3
\end{verbatim}

\normalsize

\subsection{Part g}\label{part-g}

Create a sequence from 1 to 10 by 2 and use it to print the odd rows of
the data frame.

\normalsize

\begin{Shaded}
\begin{Highlighting}[numbers=left,,]
\NormalTok{odd.rows }\OtherTok{=} \FunctionTok{seq}\NormalTok{(}\AttributeTok{from=}\DecValTok{1}\NormalTok{, }\AttributeTok{to=}\DecValTok{10}\NormalTok{, }\AttributeTok{by=}\DecValTok{2}\NormalTok{) }\CommentTok{\#using sequential vector}
\NormalTok{df[odd.rows,] }\CommentTok{\#using odd.rows in place of vector row number}
\end{Highlighting}
\end{Shaded}

\begin{verbatim}
  x1 x2 x3
1  8  1  8
3  5  3  3
5  2  2  3
7  1 10  1
9  6  2 10
\end{verbatim}

\normalsize

\subsection{Part h}\label{part-h}

Below, I create an empty column called \texttt{x12}. Fill in the details
of the \texttt{for(...)} loop to ensure \texttt{x12} is the product of
\texttt{x1} and \texttt{x2}.

\scriptsize

\begin{Shaded}
\begin{Highlighting}[numbers=left,,]
\NormalTok{n }\OtherTok{\textless{}{-}} \FunctionTok{nrow}\NormalTok{(df)                }\CommentTok{\# How many rows to we have to fill?}
\NormalTok{df}\SpecialCharTok{$}\NormalTok{x12 }\OtherTok{\textless{}{-}} \FunctionTok{rep}\NormalTok{(}\ConstantTok{NA}\NormalTok{, }\FunctionTok{nrow}\NormalTok{(df))   }\CommentTok{\# Create an empty column for x12}

\ControlFlowTok{for}\NormalTok{(i }\ControlFlowTok{in} \DecValTok{1}\SpecialCharTok{:}\FunctionTok{nrow}\NormalTok{(df))\{}
\NormalTok{  df}\SpecialCharTok{$}\NormalTok{x12[i]}\OtherTok{=}\NormalTok{df[i,}\DecValTok{1}\NormalTok{]}\SpecialCharTok{*}\NormalTok{df[i,}\DecValTok{2}\NormalTok{]}
\NormalTok{\}}
\NormalTok{df[,}\StringTok{"x12"}\NormalTok{]}
\end{Highlighting}
\end{Shaded}

\begin{verbatim}
 [1]  8 21 15 18  4 24 10 12 12  8
\end{verbatim}

\normalsize

\subsection{Part i}\label{part-i}

Write a function called \texttt{calculate.score(...)} that takes three
arguments representing \texttt{x1}, \texttt{x2}, and \texttt{x3}) and
returns a single value based on the formula:
\[Score = (x_1 \times 2) + x_2 - x_3\] Use your function to create a new
column \texttt{total.score} in our data frame \texttt{df}.

\normalsize

\begin{Shaded}
\begin{Highlighting}[numbers=left,,]
\NormalTok{x1 }\OtherTok{=}\NormalTok{ df[,}\DecValTok{1}\NormalTok{]}
\NormalTok{x2 }\OtherTok{=}\NormalTok{ df[,}\DecValTok{2}\NormalTok{]}
\NormalTok{x3 }\OtherTok{=}\NormalTok{ df[,}\DecValTok{3}\NormalTok{]}
\NormalTok{calculate.score }\OtherTok{=} \ControlFlowTok{function}\NormalTok{(x1, x2, x3)\{(}\DecValTok{2}\SpecialCharTok{*}\NormalTok{x1)}\SpecialCharTok{+}\NormalTok{x2}\SpecialCharTok{{-}}\NormalTok{x3\}}
\NormalTok{df }\OtherTok{=} \FunctionTok{data.frame}\NormalTok{(x1, x2, x3, }
                \AttributeTok{total.score=}\FunctionTok{calculate.score}\NormalTok{(x1,x2,x3))}
\NormalTok{df}
\end{Highlighting}
\end{Shaded}

\begin{verbatim}
   x1 x2 x3 total.score
1   8  1  8           9
2   3  7  2          11
3   5  3  3          10
4   3  6  2          10
5   2  2  3           3
6   3  8 10           4
7   1 10  1          11
8   2  6  1           9
9   6  2 10           4
10  2  4  1           7
\end{verbatim}

\normalsize

\subsection{Part j}\label{part-j}

Create a function called \texttt{evaluate.row(...)} that takes one
argument and returns ``low'' when the argument is less than 4, ``mid''
when the argument is between 4 and 7 (inclusive), and ``high'' when the
arguement is 8 or larger. Then, use \texttt{sapply(...)} to apply it to
column \texttt{x1}.

\normalsize

\begin{Shaded}
\begin{Highlighting}[numbers=left,,]
\NormalTok{evaluate.row }\OtherTok{=} \ControlFlowTok{function}\NormalTok{(i)\{}
                       \ControlFlowTok{if}\NormalTok{(i}\SpecialCharTok{\textless{}}\DecValTok{4}\NormalTok{)\{}
                        \StringTok{"low"}
\NormalTok{                       \}}
                       \ControlFlowTok{else} \ControlFlowTok{if}\NormalTok{(i}\SpecialCharTok{\textgreater{}=}\DecValTok{4} \SpecialCharTok{\&}\NormalTok{ i}\SpecialCharTok{\textless{}=}\DecValTok{7}\NormalTok{)\{}
                        \StringTok{"mid"}
\NormalTok{                       \}}
                       \ControlFlowTok{else}\NormalTok{\{ }
                       \StringTok{"high"}
\NormalTok{                       \}}
\NormalTok{\}}
\FunctionTok{sapply}\NormalTok{(x1, evaluate.row)}
\end{Highlighting}
\end{Shaded}

\begin{verbatim}
 [1] "high" "low"  "mid"  "low"  "low"  "low"  "low"  "low"  "mid"  "low" 
\end{verbatim}

\normalsize

\subsection{Part k}\label{part-k}

Did we need to use loops or functions in (h.)-(j.)? That is, can we use
vectorization to attain the same results in 1 line each? Where it is
possible, write the line of code. Where it is not, explain why.

I'm choosing peace and joy rn.

\section{Complete Tasks for One
Recipe}\label{complete-tasks-for-one-recipe}

\subsection{Part a}\label{part-a-1}

Install and load the \texttt{rvest} package (Wickham 2025).

\scriptsize

\begin{Shaded}
\begin{Highlighting}[numbers=left,,]
\CommentTok{\#install.packages("rvest")}
\FunctionTok{library}\NormalTok{(}\StringTok{"rvest"}\NormalTok{)}
\end{Highlighting}
\end{Shaded}

\normalsize

\subsection{Part b}\label{part-b-2}

Load the html of the \texttt{website/KimchiGrilledCheese.html} using the
\texttt{read\_html()} function and save the result to an object called
\texttt{recipe.item}. We will use the Kimchi Grilled Cheese recipe as
our prototype and extend this workflow to all recipes, so try to be as
general as possible. If you do look at it, you'll notice it contains the
html we saw in the developer tools in class. \textbf{Hint}: You can use
\texttt{read\_html(...)} like \texttt{read.csv(...)}. Don't forget you
can use the documentation to help use it.

\normalsize

\begin{Shaded}
\begin{Highlighting}[numbers=left,,]
\NormalTok{recipe.item }\OtherTok{=} \FunctionTok{read\_html}\NormalTok{(}\StringTok{"website/KimchiGrilledCheese.html"}\NormalTok{)}
\NormalTok{recipe.item}
\end{Highlighting}
\end{Shaded}

\begin{verbatim}
{html_document}
<html xmlns="http://www.w3.org/1999/xhtml" lang="en" xml:lang="en">
[1] <head>\n<meta http-equiv="Content-Type" content="text/html; charset=UTF-8 ...
[2] <body class="nav-fixed quarto-light">\n\n<div id="quarto-search-results"> ...
\end{verbatim}

\normalsize

\subsection{Part c}\label{part-c-1}

Open the html file in a web browser and open developer tools. In
chrome-based browsers, you can do this by pressing the three verticle
dots in the upper-right corner, clicking ``more tools'', then
``developer tools''.

Find the name of the Ingredients section of the website and pull the
HTML of the function using \texttt{html\_element(...)} and save the
results to an object called \texttt{ingredients.section}. \textbf{Hint}:
You can ask for elements by id using a preceeding ``\#''. See
\texttt{?html\_element(...)} for a helpful example.

\normalsize

\begin{Shaded}
\begin{Highlighting}[numbers=left,,]
\NormalTok{ingredients.section }\OtherTok{=} \FunctionTok{html\_element}\NormalTok{(recipe.item, }\StringTok{"\#ingredients"}\NormalTok{)}
\NormalTok{ingredients.section}
\end{Highlighting}
\end{Shaded}

\begin{verbatim}
{html_node}
<section id="ingredients" class="level2">
[1] <h2 class="anchored" data-anchor-id="ingredients">Ingredients</h2>
[2] <ul>\n<li>1/8 tsp chili flakes</li>\n<li>4 garlic cloves</li>\n<li>5 oz s ...
\end{verbatim}

\normalsize

\subsection{Part d}\label{part-d-1}

Now, we want to obtain all of the itemized items. Find the element type
the individual ingredients and pull all of them from
\texttt{ingredients.section} using \texttt{html\_elements(...)} (note
the added s) and save the results to an object called
\texttt{ingredients}. \textbf{Hint}: You can ask for elements by type by
simply specifying the tag (e.g., ``p'' for paragraph). See
`?html\_element for a helpful example.

\normalsize

\begin{Shaded}
\begin{Highlighting}[numbers=left,,]
\NormalTok{ingredients }\OtherTok{=} \FunctionTok{html\_elements}\NormalTok{(ingredients.section, }\StringTok{"li"}\NormalTok{)}
\end{Highlighting}
\end{Shaded}

\normalsize

\subsection{Part e}\label{part-e-1}

Similar to Part c.~Find the name of the Instructions section of the
website and pull the HTML of the function using
\texttt{html\_element(...)} and save the results to an object called
\texttt{instructions.section}.

\normalsize

\begin{Shaded}
\begin{Highlighting}[numbers=left,,]
\NormalTok{instructions.section }\OtherTok{=} \FunctionTok{html\_element}\NormalTok{(recipe.item, }\StringTok{"\#instructions"}\NormalTok{)}
\NormalTok{instructions.section}
\end{Highlighting}
\end{Shaded}

\begin{verbatim}
{html_node}
<section id="instructions" class="level2">
[1] <h2 class="anchored" data-anchor-id="instructions">Instructions</h2>
[2] <ol type="1">\n<li>In a frying pan, heat 2 tablespoons olive oil with chi ...
\end{verbatim}

\normalsize

\subsection{Part f}\label{part-f-1}

Now, we want to obtain all of the enumerated items. Find the element
type the individual instructions and pull all of them from
\texttt{instructions.section} using \texttt{html\_elements(...)} (note
the added s) and save the results to an object called
\texttt{instructions}.

\normalsize

\begin{Shaded}
\begin{Highlighting}[numbers=left,,]
\NormalTok{instructions }\OtherTok{=} \FunctionTok{html\_elements}\NormalTok{(instructions.section, }\StringTok{"li"}\NormalTok{)}
\NormalTok{instructions}
\end{Highlighting}
\end{Shaded}

\begin{verbatim}
{xml_nodeset (5)}
[1] <li>In a frying pan, heat 2 tablespoons olive oil with chili flakes.</li>
[2] <li>Add spinach and cook until wilted.</li>
[3] <li>Mix gochujang and mayonnaise in a small bowl and spread on one side o ...
[4] <li>Build sandwiches by layering spinach, kimchi, and cheese between two  ...
[5] <li>Fry in butter until golden brown on both sides.</li>
\end{verbatim}

\normalsize

\subsection{Part g}\label{part-g-1}

Find the class of the recipe image and pull the HTML using
\texttt{html\_element(...)} and save the results to an object called
\texttt{image.element}. \textbf{Hint}: You can ask for elements by class
using a preceeding ``.''. See \texttt{?html\_element(...)} for a helpful
example. Further, note that HTML elements may have more than one class
(separated by a space). When that is the case, you need to choose one.

\normalsize

\begin{Shaded}
\begin{Highlighting}[numbers=left,,]
\NormalTok{image.element }\OtherTok{=} \FunctionTok{html\_element}\NormalTok{(recipe.item,}\StringTok{"\#quarto{-}figure"}\NormalTok{)}
\end{Highlighting}
\end{Shaded}

\normalsize

\subsection{Part h}\label{part-h-1}

Use the \texttt{html\_attr(...)} function to pull the source link
(``src''). Then, use \texttt{paste(...)} to prepend the source link with
``website/'' so we have the full link. \textbf{Note}: This would be like
adding ``https://www.website.com'' to get the absolute link.

\normalsize

\begin{Shaded}
\begin{Highlighting}[numbers=left,,]
\FunctionTok{html\_attr}\NormalTok{(image.element, }\StringTok{"src"}\NormalTok{)}
\end{Highlighting}
\end{Shaded}

\begin{verbatim}
[1] NA
\end{verbatim}

\begin{Shaded}
\begin{Highlighting}[numbers=left,,]
\FunctionTok{paste}\NormalTok{(}\FunctionTok{html\_attr}\NormalTok{(image.element, }\StringTok{"src"}\NormalTok{))}
\end{Highlighting}
\end{Shaded}

\begin{verbatim}
[1] "NA"
\end{verbatim}

\normalsize

\subsection{Part i}\label{part-i-1}

Now that we have all the things we need, let's try to print the recipe.
Below, I have written code to print from the objects. One by one, remove
\texttt{\#\textbar{}\ eval:\ false} from the YAML header and add
\texttt{\#\textbar{}\ echo:\ false} and
\texttt{\#\textbar{}\ results:\ \textquotesingle{}asis\textquotesingle{}},
and test. Let me know if you're stuck!

\textbf{Image}

\normalsize

\begin{figure}[H]

\caption{Kimchi Grilled Cheese}

{\centering \includegraphics[width=0.5\linewidth,height=\textheight,keepaspectratio]{NA.pdf}

}

\end{figure}%

\normalsize

\textbf{Ingredients}

\normalsize

\begin{itemize}
\tightlist
\item
  1/8 tsp chili flakes
\item
  4 garlic cloves
\item
  5 oz spinach
\item
  1 tsp gochujang
\item
  3 tsp mayonnaise
\item
  2/3 cup kimchi
\item
  4 slices of bread
\item
  2/3 cup cheddar cheese
\item
  4 tablespoons everything seasoning.
\end{itemize}

\normalsize

\textbf{Instructions}

\normalsize

\begin{enumerate}
\def\labelenumi{\arabic{enumi}.}
\tightlist
\item
  In a frying pan, heat 2 tablespoons olive oil with chili flakes.
\item
  Add spinach and cook until wilted.
\item
  Mix gochujang and mayonnaise in a small bowl and spread on one side of
  each slice of bread. Sprinkle with everything seasoning and press it
  into the bread.
\item
  Build sandwiches by layering spinach, kimchi, and cheese between two
  slices of bread.
\item
  Fry in butter until golden brown on both sides.
\end{enumerate}

\normalsize

\section{Complete a Full Menu!}\label{complete-a-full-menu}

Open \texttt{Menu.qmd} and add code to complete the following.

\begin{enumerate}
\def\labelenumi{\arabic{enumi}.}
\tightlist
\item
  Randomly select three dinner recipes and one breakfast recipe at
  random. \textbf{Hint:} Use the \texttt{sample(...)} function.
\item
  Pull the image, ingredients, and instructions for each selected
  recipe.
\item
  Combine all of the ingredients into a grocery list on the first page.
\item
  Print the image, ingredients, and instructions for each receipe on the
  subsequent pages.
\end{enumerate}

When you render this document, no code should be visible. Instead, you
should see a five-page document as described above.

\section{Describe your work!}\label{describe-your-work}

\subsection{Why a fake website?}\label{why-a-fake-website}

Read
\href{Is\%20Web\%20Sraping\%20Legal?}{https://www.scrapingbee.com/blog/is-web-scraping-legal/}.
Originally, I conceived doing this with recipes from Purple Carrot (my
favorite subscription service). Being a large company, their terms of
service are \emph{very} long and precluded us from copying their
recipes. We also checked a few smaller recipe websites we like and even
they had terms of service that restricted automated collection of data.

\subsection{Conditionals}\label{conditionals}

Did you use conditional statements in your code? If yes, how. If not,
are there places you could have used them but did something else?

\subsection{Loops}\label{loops}

Did you use loops in your code? If yes, how. If not, are there places
you could have used them but did something else?

\subsection{Functions}\label{functions}

Did you use functions in your code? If yes, how. If not, are there
places you could have used them but did something else?

\section*{References}\label{bibliography}
\addcontentsline{toc}{section}{References}

\phantomsection\label{refs}
\begin{CSLReferences}{1}{0}
\bibitem[\citeproctext]{ref-rvest}
Wickham, Hadley. 2025. \emph{Rvest: Easily Harvest (Scrape) Web Pages}.
\url{https://doi.org/10.32614/CRAN.package.rvest}.

\end{CSLReferences}




\end{document}
